% Lecture template for GGI Dark Matter School notes
% Copy this file and customize for each lecture

\section{Lecture Title}
\label{sec:lecture}

% Lecture metadata
\textbf{Lecturer:} Name \\
\textbf{Date:} Date \\
\textbf{Topics:} Brief overview

\subsection{Introduction}

Brief introduction to the lecture topic.

\subsection{Main Content}

\subsubsection{Key Concept 1}

Explanation of the first key concept.

\begin{definition}[Important Term]
Define important terms here.
\end{definition}

\begin{theorem}[Main Result]
State important theorems.
\end{theorem}

\begin{proof}
Proof goes here.
\end{proof}

\subsubsection{Key Concept 2}

More content here.

% Example equation
\begin{equation}
    \label{eq:important}
    E = mc^2
\end{equation}

Refer to equation~\eqref{eq:important} using the label.

% Example figure
\begin{figure}[ht]
    \centering
    % \includegraphics[width=0.7\textwidth]{figures/your-figure.pdf}
    \caption{Caption describing the figure.}
    \label{fig:example}
\end{figure}

Refer to Figure~\ref{fig:example} in the text.

\subsection{Examples}

\begin{example}
Concrete example demonstrating the concepts.
\end{example}

\subsection{Discussion}

Discussion of implications and connections to other topics.

\subsection{Questions and Open Problems}

\begin{itemize}
    \item Question 1
    \item Question 2
    \item Open problem to investigate
\end{itemize}

\subsection{References for Further Reading}

Key papers and resources related to this lecture~\cite{author2024}.
